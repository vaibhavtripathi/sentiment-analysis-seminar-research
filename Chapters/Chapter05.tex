\chapter*{Conclusions}

Sentiment analysis is an emerging field with a thrilling potential. It, along with the power of social networks, can harness a great deal of unseen and useful information.
Since this information is coming from people, it allows the providers to be in touch with their consumers at all times, understanding their demands and grievances.
Sentiment analysis systems, thus, seek to create a world where the voice of the people is given a status as never before and their needs are always understood, wherever
and howsoever, they may express it.

\vspace{8mm}

It being established that sentiment analysis systems can achieve a lot, some doubt the future of research in this direction, saying that sentiment systems can never 
achieve an accuracy at which they can reliable enough to be instrumental to global change. The realization to be made here is that the gold standard here is not a full 100\%.

\vspace{8mm}

Sentiment is humanly. So, the performance of sentiment analysis systems has to be measured against a human's ability to do the same. Incidentally, there are studies which
show that there is an agreement of about 70\% among humans regarding sentiment on any subject. This is called as human concordance. So, this is the accuracy that sentiment systems 
target, which is very much achievable.


\vfill